\documentclass[brazil,twocolumn]{svjour3}
\usepackage[portuguese]{babel}

\author{Rafael Stoffalette João\\
\texttt{rafaelcompp@gmail.com}}
\title{ PREVISÃO DE MERCADO PARA AÇÕES EM BOLSA DE VALORES BASEADO EM TÉCNICAS DE INTELIGÊNCIA ARTIFICIAL
}
\date{\today}

\begin{document}
\maketitle

O Mercado de ações e a bolsa de valores despertam interesses em grande parte da sociedade, muitos ja são invesitdores, outros tem vontade de investir mas tem medo de manobras arriscadas e, consequentemente, obter prejuízo por não conhecerem o ambiente em que estão analisando. Com intuito de popularizar esta, e muitas outras, áreas, técnicas computacionais e novas tecnologias são testadas e implantadas e tem demonstrado resultados mais que satisfatórios em grande parte das pesquisas realizadas. 
A falta de confiança em uma atitude tomada por uma máquina é um fator quase que desconsiderado em tempos atuais, pois a cada dia mais, o avanço da tecnologia permite que algoritmos bem elaborados e programas desenvolvidos provem que a evolução depende, quase que, diretamente desse avanço  e de tecnicas computacionais.
O proposito deste estudo é, prioritariamente, aplicar conceitos de inteligência artificial baseando-se no funcionamento de uma rede neural artificial com multiplos neurônios em sua camada oculta, realimentada e treinada por um algoritmo backpropagation  no mercado de ativos da bolsa de valores e assim, considerando-se um bom conjunto histórico de cotações, prever qual será a próxima cotação para a ação analisada.
Trata-se então, de uma ferramenta que visa auxiliar possíveis investidores a ter mais segurança na escolha das ações, tomar decisões e efetuar manobras no ambito de obter melhores resultados minimizando os riscos e maximizando os resultados.

Palavras-chave: bolsa de valores, ações, inteligência artificial, rede neural, predição, aprendizado de máquina.

Introdução
Este documento tem por finalidade viabilizar um estudo que interprete informações coletadas de fontes econômicas da internet, trate-as e as utilize de forma a alimentar uma rede neural artificial (RNA) implementada seguindo teorias de inteligência artificial e que assim, a treine usando um algoritmo backpropagation e a capacite para retornar resultados satisfatórios no sentido de prever a oscilação existente no mercado de valores das ações listadas em uma bolsa de valores.
A estruturação do documento segue uma linha de esclarecimento de conceitos e teorias que são de essencial entendimento para a analise do objetivo, sendo assim a seção primeira deste descreve a problemática do assunto, a justificativa da necessidade do estudo para essa área e a motivação que leva ao estudo. Bem como uma breve introdução a conceitos voltados ao mercado de valores.
Já na seção subsequente, é abordado todo o tema básico sobre uma bolsa de valores e seu funcionamento englobando termos técnicos e descrição sequencial de eventos que ocorrem para que a bolsa opere normalmente.
Em seguida, o documento apresenta seu foco principal de estudo, as ações que a bolsa de valores negocia, seus conceitos, tipos, como identificá-las em uma fonte e informação e como é feita a negociação das ações pelos investidores que se utilizam dos serviços de uma corretora de ações.
Uma abordagem básica explanatória é abordada na seção seguinte do documento explicando alguns conceitos fundamentais sobre inteligência artificial para o entendimento das técnicas utilizadas no decorrer do estudo. Mais precisamente, o documento foca nas redes neurais artificiais, seu funcionamento, sua finalidade e como é feito seu treinamento para que possa exibir resultados satisfatórios para o estudo. Também há nesta seção uma definição sobre redes e algoritmos de predição que poderão colaborar muito no desenvolvimento do sistema proposto pelo estudo.
Prosseguindo, o documento aborda estudos passados que focaram no mesmo, ou semelhante, tema deste estudo e seus resultados obtidos. Muitos estudos já foram feitos nesta área, muitos com resultados positivos, porem poucos apresentaram taxas de erro consideráveis, e assim pode-se analisar melhor e escolher qual linha de raciocínio seguir. Consequentemente, assim pode-se confrontar resultados obtidos com os esperados pelo sistema proposto.
O document ainda apresenta as considerações finais do estudo com informações referentes ao que se espera que o sistema retorne. Sabe-se que é uma tarefa difícil prever a movimentação das ações no mercado de valores e que qualquer fator externo pode ser influenciador direto sobre o valor da ação, portanto é abordado nesta seção as possibilidades e as restrições que são encontradas no desenvolvimento do sistema.
Por fim, o documento apresenta todas as tentativas utilizadas, resultados obtidos, considerações e fontes de informações que foram consultadas para sua criação, incluindo livros, artigos, estudos realizados anteriormente que abordam temas semelhantes e sites da internet que abordam tanto conceitos de mercado de ações na bolsa de valores quanto técnicas e conceitos que serão utilizados na implementação do sistema proposto pelo estudo.

Motivação
É recorrente discussões sobre investimentos e formas de ganhar dinheiro que enfatizem a possibilidade de bons resultados em investimentos no mercado de ações da bolsa de valores devido ao crescimento adquirido desse setor no mercado financeiro com o avanço da globalização. Tão quanto é normal o desconhecimento detalhado das pessoas sobre o assunto e possibilidades de atuações. Atualmente o mercado de ações, é uma forte fonte de financiamento para as empresa e também, um meio importante de aquisição financeira para pessoas comuns.
Também conhecido é que esse ramo de atividades não é tão estável e torna investidores, potencialmente inclusive, receosos e cautelosos. Instabilidade essa que não é possível de ser estimada e superada devido a fatores políticos, econômicos, entre outros, também conhecidos como agentes externos que influenciam direta e indiretamente sobre ações de empresas que compõem o mercado da bolsa de valores.
Outro fator importante a se ressaltar é a dificuldade existente à acionistas amadores para controlar suas ações e, ao mesmo tempo, pesquisar sobre novas possíveis boas escolhas. Tanto pela quantidade de informações que uma pessoa deve processar, quanto pela escolha certa de fontes que tragam informações úteis e reais para análise.
Atualmente a base mais sólida para aquisição de informações que ajudem escolhas e passos no mercado de valores é a estatística, que apresenta dados reais e comportamentos que as ações, empresas e investidores tiveram em situações distintas. Porém mostra dados passados e que possivelmente não ocorrerão mais, fica assim responsável por uma boa parcela das tomadas de decisões dos acionistas as informações sobre o mundo e uma estimativa pessoal conhecida como “feeling”.
Hoje, existem tecnicas e conceito que podem ampliar os conhecimentos sobre o mercado de ações e a bolsa de valores, bem como sua utilização e possibilidades de escolha. A internet por exemplo, é um ambiente totalmente propício ao crescimento e popularização do setor, assim como a inteligência artificial, que prove formas de tormar uma máquina consciente em tomadas de decisões. 
As redes neurais artificiais são algumas das técnicas mais difundidas e utilizadas da inteligência artificial, com elas pode ser possível auxiliar um usuário, com o mínimo de conhecimento necessário sobre o tema do estudo e ambiente de internet, a fazer boas escolhas em tomadas de decisões e até mesmo entender melhor como funciona o mercado e como é sensível a fatores externos que, apesar de não parecer, influenciam diretamente em seus investimentos.
Falar sobre bolsa de valores tornou-se um interesse global, que envolve desde investidores em busca de lucros até empresas que querem conquistar novos acionistas, mantendo os já existentes, afim de valorizar-se e arrecadar investimentos para a evolução da empresa. Dessa visão é possível ver a necessidade de desenvolver um sistema que possa chegar o mais próximo possível de uma previsão bem formulada (como um ser humano executaria) e obter um retorno positivo.
Desde o já é válido a ressalta sobre os riscos de não se obter o resultado esperado, fato que é possível de ocorrer tanto por uma escolha derivada de um pensamento humano quanto por uma certa influência de ferramentas como a proposta pelo estudo. Portanto, reafirma-se que a ferramentas desenvolvida é uma colaboradora na tomada de decisões, não devendo ser a única fonte de informações.
De acordo com Armano, Marchesi e Murru [1], o preço de uma ação reflete diretamente, em qualquer tempo, as informações que os investidores possuem. Ou seja, assim que uma informação nova é processada o preço da ação é alterado.
A área de invesitmentos e finanças é uma das principais áreas de aplicação das redes neurais artificiais, por tratar de um ramo em que decisões tomadas baseando-se em fontes de informações com um alto grau de variação e imprecisão [2]. As redes neurais artificials se destacam pela sua capacidade de aprendizado de padrões, ainda que mesmo modelos estatísticos prevejam tendências do mercado financeiro, as redes neurais artificiais são mais apropriadas para lidar com oscilações recorrentes.

A Bolsa de Valores
Uma bolsa de valores é um ambiente, ou por alguns especialistas definida, uma instituição administrativas de negócios onde títulos emitidos por empresas são manipulados. As companhias que são “listadas”, assim chamadas as empresas que tem suas ações negociadas na bolsa, podem ser de capital público, privado ou misto, esse quando mesclam entre seus ativos investimentos governamentais e de investimentos privados, e ambas tem seus capitais comercializados de forma eletrônica.
No Brasil, atualmente, as bolsas são organizadas sob a forma de sociedade por ações (S/A), reguladas e fiscalizadas pela CVM(Comissão de valores mobiliários).
A bolsa de valores tem como principal funcionalidade, proporcionar de forma transparente e líquida um ambiente para que valores de empresas possam ser comercializados. Somente por meio de corretoras é que investidores podem ter acesso aos sistemas de negociação e efetuarem transações de compra e venda de valores.
Para que suas ações possam ser negociadas na bolsa de valores, uma companhia deve ser aberta, isto é, o público em geral detém suas ações e não uma parcela restrita. E ainda, seguir instruções da CVM, além de cumprir uma série de normas e regras definidas pela própria bolsa.
No website do portal do investidor [3] é possivel encontrar pesquisas que mostram que o mercado de capitais é mais eficientes em países que possuem bolsas de valores bem estruturadas.
Ao contrário do que se imagina, esse mercado não proporciona benefícios apenas aos investidores ativos da bolsa de valores, mas indiretamente atua na melhoria de qualidade de vida de todos.
Dos benefícios proporcionados, destacam-se os seguintes:
- As companhias encontram nas bolsas de valores um ambiente propício para levantar capital pela compra e venda de suas ações, e assim, expandir suas atividades, qualidade de produtos, melhor poder aquisitivo para a população;
- Quando uma pessoa decide entrar no mercado de ações de companhias, inconscientemente ela, coopera para uma realocação racional de recursos, isto é, ao investir em um segmento, a empresa beneficiada pode expandir movimentando tanto o seu negócio quanto outros ramos de atividades. Esta atividade gera novos beneficiários de vários setores da economia, resultando num crescimento cooperativo e rápido;
- O fluxo de capitais estimula o crescimento das companhias devido ao crédito investido nas empresas, gera-se novos empregos direta e indiretamente;
- Consequentemente associada a demanda cada vez maior de acionistas de uma companhia, o governo impõe regras mais rígidas à bolsa de valores e isto impulsiona as empresas a evoluir administrativamente. É comum dizer que companhias abertas são mais bem administradas que as fechadas( companhias que ações não são negociadas, ou que pertencem a familiares ou a um grupo restrito de investidores);
- Por ser aberto, qualquer pessoa que queira participar do mercado de valores tem essa possibilidade, e os investimentos em ações não requerem uma quantia expressiva de seus investidores. Um pequeno investidor pode adquirir uma pequena quantidade de ações e assim como todos os demais investidores, fazer parte dos rendimentos associados;
- A bolsa de valores torna-se um termômetro da economia, as ações das companhias listadas por ela oscilam diretamente influenciadas pelos fatos ocorridos no mundo e forças do mercado. Qualquer informação propagada pode mover os valores de ações da bolsa, assim analisar os índices de ações torna-se um bom fator indicativo de tendências da economia;
- Os governo pode usar a bolsas de valores ao emprestar dinheiro para a iniciativa privada a fim de financiar projetos para camadas sociais mais inferiores. Geralmente, esses tipos de projetos necessitam de grande volume de recursos, e que as empresas não teriam condições de levantar sozinhas. Os governos, para levantarem recursos, emitem títulos públicos que podem ser negociados nas bolsas de valores. O levantamento de recursos privados, por meio da emissão de títulos, elimina a necessidade (pelo menos no curto prazo) dos governos sobretaxarem seus cidadãos.

3.2 – Como funciona a bolsa de valores
O funcionamento da bolsa de valores é simples de se entender e baseia-se nas atividades descritas no trecho a seguir. Quando uma empresa necessita de capital para investir em prol de crescimento e os juros que os bancos estabelecem para empréstimos não são tão agradáveis, então ela decide lançar ações ao público, isto é, abrir seu capital para que investidores possam deter parte de seu capital e em troca forneçam dinheiro para seus investimentos.
Ao abrir o capital, são feitas várias análises e cálculos para se estabelecer o preço correto de cada ação e a qual a quantidade de ações que a companhia pode oferecer no mercado. É o chamado “mercado primário” no qual corretoras e bancos permitem que seus clientes comprem estas ações.
Pagos os devidos impostos, o montante arrecadado com a venda das ações é de uso exclusivo da empresa e como forma de retribuição à isenção de impostos, 25% do lucro líquido obtido pela empresa deve ser dividido à seus acionistas.
Saindo do mercado primário os detentores das ações podem querer vende-las em um determinado tempo, nesse momento começa a atuação da bolsa de valores. A iniciativa atrais novos investidores que, utilizando-se de uma corretora de valores credenciada à companhia, injetam seu dinheiro comprando ações. Quanto mais interessados em comprar uma ação, mas ela tende a se valorizare se destacar entre os acionistas. Esse interesse é mais especulativo do que real, ou seja, baseia-se na esperança de que a ação irá se valorizar e, assim, revendê-la por um preço maior. O inverso também é verdadeiro, nos casos em que muitos querem se desfazer das ações e ninguém quer comprá-las, a tendência é que os detentores das ações diminuam o preço delas para atrair compradores e assim, consequentemente a ação desvaloriza. A oscilação da bolsa pode ser definida basicamente como a flutuação natural da compra e venda de ações.

3.3– Ações e valor de mercado
Ações são títulos de renda variável emitidas por sociedades anônimas e representam a menor fração em que se divide o capital de uma empresa. Assim define-se uma empresa como uma sociedade por ações, aberta ou fechada. Um detentor de ações de uma empresa é um coproprietário da sociedade anônima a qual é associado, portanto tem o direito de gozar de parte dos resultados obtidos pela companhia correspondente a seu investimento.
Ações podem ser de dois tipos, ordinárias ou preferenciais. As primeiras permitem que seus detentores tenham direito ao voto nas assembleias gerais de acionistas, votos de deliberação referente à administração da empresa, balanço e futuros investimentos. As ações ordinárias, visualmente, são distinguidas por apresentarem o sufixo “ON” em seu nome. Já as ações preferenciais não permitem que seus detentores votem, mas atribuem o direito a seus detentores de receber dividendos com prioridade sobre os detentores de ações ordinárias e também de receber o capital investido em caso de liquidação da companhia. Ações preferenciais possuem um sufixo “PN” que as distingue também.
Além dos sufixos descritos neste mesmo capítulo, uma ação é reconhecida por apresentar um código da empresa que representa seguido do código que descreve o tipo de ação. Por exemplo, EMBRATEL PN pode ser representada pelo código EBTP4, onde o dígito 4 define seu tipo, preferencial (PN), assim como o dígito 3 define as ações ordinárias (ON).
Ações PN são ainda, subdivididas por classes que hierarquicamente define a ordem de pagamento de dividendos e reembolso. PNA, PNB e PNC são, respectivamente, representadas pelos inteiros 5, 6 e 7.
A popularidade de uma ação no mercado é baseada em sua qualidade de negócio, isto é, a facilidade que o detentor tem de converter suas ações em moeda real. Esta definição é conhecida no mercado de valores pelo termo “liquidez de uma ação”. Ações de alta liquidez, são referentes a grandes empresas com alta reputação e prestígio, também conhecidas como “blue chips”. Estas ações são conhecidas pela sua estabilidade de valor e valorização a longo prazo garantida, consequentemente são as que detém o maior valor de mercado.
Em sequencia, existem as ações de segunda linha, caracterizadas pela sua sensibilidade ao mercado, sua queda sempre precede à queda das blue chips assim como, consequentemente,  sua valorização somente ocorre após a valorização das blue chips.
Por fim, existem as ações de terceira linha. Estas possuem liquidez consideravelmente baixa e são caracterizadas por serem de empresas pequenas e com baixa reputação, o que não indica que são de menor qualidade.
O valor das ações é definido pela sua cotação, alvo de importância para este estudo. As cotações podem ser de abertura, a primeira cotação do dia, mínima, quando atinge o menor valor durante toda atividade do dia, máxima que, inversamente proporcional à mínima, define a menor cotação do dia. Também é utilizada a cotação média que faz uma média de todas as cotações que a ação possuiu no dia e por fim a cotação de fechamento, que representa o ultimo valor que a ação atingiu durante o dia de atividade.
Fazendo relação com o mundo real, as ações podem ser convertidas em valor de moeda do pais, a essa definição atribui-se o termo “índice”, que além de representar a conversão para moeda real, também serve para medir a variação das
ações da bolsa e abstrai uma visão geral sobre o mercado de valores para os acionistas. No Brasil, o principal índice de ações é o iBovespa que acompanha a evolução das cotações das ações e é um forte indicador de desempenho do mercado de ações em geral, porém existem vários outros índices que não são tão utilizados quanto este. Por exemplo, pode-se citar índices como IBRX, IBRX-50, IVBX-2, IGC, dentre outros vários.
O índice Bovespa é o valor atual, em moeda corrente de uma carteira teórica de ações constituída em 02/01/1968, a partir de uma aplicação hipotética. Supõe-se não ter sido efetuado nenhum investimento adicional desde então, considerando-se somente os ajustes efetuados em decorrência da distribuição de proventos pelas empresas emissoras( tais como inversão de dividendos recebidos e do valor apurado com a venda dos direitos de subscrição, e manutenção em carteira das ações recebidas de bonificação). Dessa forma, o índice reflete não apenas as variações dos preços das ações, mas também o impacto da distribuição de proventos.
Como já citado anteriormente, uma pessoa só pode se tornar um acionista se utilizar de intermediários financeiros, mais conhecidos como corretoras de ações. Estas empresas são portadoras de uma gama considerável de informações sobre as empresas listadas pela bolsa de valores e o mercado e são de fácil acesso a todos, inclusive pela internet hoje é possível que qualquer pessoa se informe e negocie ações com outros investidores sem ter que enfrentar problemas como linhas de telefone ocupadas, tempo de espera dos correios ou transtorno de locomoção física. O “home broker”, sistema digital que assemelha a uma rede social, é a forma mais fácil de um investidor negociar sua ações, através dele o acionista pode enviar ordens de compra e venda pelo site da corretora a qual é cadastrado. Outra forma de negociação é o pregão online no próprio site da Bovespa, onde são exibidos em os valores das ações negociadas em tempo real.
A negociação de ações pela internet é a maior responsável pelo fim, em 2005, do pregão viva voz ( Figura1), onde representantes das corretoras associadas à Bovespa negociavam em voz alta suas ações em um espaço reservado à essa atividade e que demonstra uma total confusão a quem não é profundo entendedor do mercado.
Encontrar um padrão na oscilação das ações é uma tarefa que pode ser considerada impossível, já que o mercado de valores é muito sensível a diverso aspectos, assim alguns investidores seguem fontes de informações das mais variadas bases. Na própria terminologia do mercado de ações há dois tipos de análises que cooperam muito na formação de opinião para aquisição de novas ações pelos investidores, são as análises fundamentalistas e análises técnicas.
Na análise fundamentalista o foco do estudo é a causa do movimento nos preços das ações, já na análise técnica os investidores focam no efeito que é causado quando há qualquer tipo de movimento sobre os valores das ações do mercado.
A análise técnica segue a teoria descrita por Chales H. Dow, conhecido por ser o fundador da escola técnica. Sua teoria, conhecida como “Teoria Dow”[4], afirma que o que aconteceu ontem pode determinar o que acontecerá hoje e a configuração gráfica dos preços tende a se relacionar com a direção que eles tomarão no futuro”, ou seja, antes de qualquer oscilação que ocorre no mercado de ações, positiva ou negativa, houve algum acontecimento que sinalizou que algo ocorreria.
Ainda seguindo a teoria de Dow, o mercado pode ser dividido em três tendências, primária que possui retornos a longo prazo, secundária que apresentam resultados em reversão à tendência primária e terciária, que apresenta tendências de curto prazo.
Por outro lado a análise fundamentalista, mais utilizada pelos analistas para acompanhar as oscilações do mercado e prever tendências, afirma que a melhor forma de se calcular o valor justo para uma empresa e suas ações está diretamente relacionada à sua capacidade de gerar lucros no futuro. O objetivo fundamental da análise fundamentalista é avaliar as alternativas de investimento a partir do processamento das informações obtidas diretamente das empresas, ou seja, analisar dados atuais sobre a empresa e sobre o ramo de atuação que ela opera.
Dessa forma espera-se que o sistema proposto se comporte analisando movimentações passadas de uma ação e assim identificar um padrão que possa ser aplicado no próximo período da cotação e que chegue o mais próximo possível do real caminho seguido por esta oscilação.

5.1 – Redes Neurais Artificiais (RNA)
“A Inteligência Artificial é um termo que abrange muitas definições” [5]. Mas basicamente pode ser considerada uma área da ciência da computação voltada para a busca de métodos que façam uma máquina reagir a situações como se fosse um ser humano, representar seu pensamento com criatividade, agilidade e essencialmente e, acima de tudo, aprendizagem. 
São duas linhas de pensamento, a inteligência artificial forte que tem uma abordagem no qual admite-se que a máquina pode raciocinar e resolver problemas de forma consciente como um ser humano. Já na inteligência artificial fraca, admite- se que o computador não é capaz de pensar efetivamente e sim, agir como se fosse inteligente, mas não tem consciência, apenas simula. Baseia-se na máquina de Turing.
Redes neurais artificiais são sistemas, também definidos como técnicas, computacionais estruturados numa aproximação à computação baseada a ligações que simulam o funcionamento de um neurônio biológico e são capazes de prever sistemas não lineares, o que torna a sua aplicação no mercado financeiro bastante eficiente. Apresentam um modelo matemático inspirado na estrutura neural de organismos inteligentes e que adquirem conhecimento através da experiência, esse modelo é conhecido como neurônio artificial (Figura 3). Nós fazem analogia a neurônios do cérebro e são interligados para formar uma rede de nós, daí o termo rede neural.
Uma rede neural, segundo Haykin[6], pode ser definida como “um processador maciçamente paralelamente distribuído, constituído de unidades de processamento simples, que têm a propensão natural para armazenar conhecimento experimental e torná-lo disponível para uso. Ela assemelha-se ao cérebro em dois aspectos: (1) o conhecimento é adquirido pela rede a partir de seu ambiente através de um processo de aprendizagem; (2) forças de conexão entre neurônios (os pesos sinápticos) são utilizadas para armazenar o conhecimento adquirido”.
Uma rede neural artificial é composta por um conjunto de unidades de processamento de funcionamento simples, os neurônios, e que são conectadas por canais de comunicação com um determinado peso associado. O aprendizado em RNAs é feito de forma interativa e evolutiva e está normalmente associado à capacidade de as mesmas adaptarem seus parâmetros como consequência da sua interação com o meio externo.
No esquema da figura 3 as entradas do neurônio correspondem ao vetor de entradas X com dimensão m e para cada uma das entradas há um peso associado w. A soma das entradas associadas a seus pesos é que caracteriza a saída linear u do neurônio que aplicada uma função de ativação, ou limiar, f(u) obtém-se a saída do neurônio, também chamada de saída de ativação, y. 
Um neurônio artificial apresenta três elementos básicos, peso sináptico, função soma e função de transferência. Os pesos sinápticos são responseveis por armazenar as informações e também implicam no efeito que a saída de um neurônio causará na entrada do próximo neurônio.

\end{document}